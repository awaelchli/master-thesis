\thispagestyle{empty}
\vspace{8cm}
\noindent
{\centerline {\bf \large Abstract}}
\vspace{1cm}

\noindent
In computer vision, \emph{Visual Odometry} is the problem of recovering the camera motion from a video.
It is related to \emph{Structure from Motion}, the problem of reconstructing the 3D geometry from a collection of images.
Decades of research in these areas have brought successful algorithms that are used in applications like autonomous navigation, motion capture, augmented reality and others.
Despite the success of these prior works in real-world environments, their robustness is highly dependent on manual calibration and the magnitude of noise present in the images in form of, e.g.\@, non-Lambertian surfaces, dynamic motion and other forms of ambiguity.
This thesis explores an alternative approach to the Visual Odometry problem via \emph{Deep Learning}, that is, a specific form of machine learning with artificial neural networks.
It describes and focuses on the implementation of a recent work that proposes the use of \emph{Recurrent Neural Networks} to learn dependencies over time due to the sequential nature of the input.
Together with a convolutional neural network that extracts motion features from the input stream, the recurrent part accumulates knowledge from the past to make camera pose estimations at each point in time.
An analysis on the performance of this system is carried out on real and synthetic data.
The evaluation covers several ways of training the network as well as the impact and limitations of the recurrent connection for Visual Odometry.
