\chapter{Introduction}

	\section{Relative Camera Pose}
		Equations to transform pose in world coordinates to relative pose with respect to the coordinate frame of the first camera.

	\section{Rotation Metrics}
	\cite{huynh2009metrics}
	
	\section{Two-Frame Structure from Motion}
	
	\section{Multiview Structure from Motion}
	
	\section{Camera Pose and Transformations}
		The camera pose, also known as the \emph{extrinsic parameters}, is a combination of position and orientation.
		Together, they define a coordinate transformation from the camera coordinate frame to the world coordinate frame.
		Such a transformation is commonly represented by a translation vector $\vectr{t} \in \mathbb{R}^3$ and a rotation matrix $\matr{R} \in SO(3)$. 
		Together, they describe a rigid transformation which can also be written as a $4 \times 4$ matrix of the form
		\begin{equation}
			\matr{T} = 
			\begin{bmatrix}
				\matr{R} 	& \vectr{t} \\
				0 			& 1
			\end{bmatrix} 
			\in SE(3).
		\end{equation}
		\todo{Insert a figure showing: world coordinate frame, camera coordinate frame, transformation matrix}
		
		In Structure from Motion, there is not only one camera, but many cameras with different transformations.
		This is also the case in this work, but the sequence of transformations is ordered according to the path of the camera motion. 
		Specifically, we deal with a sequence of rotations $\matr{R}_1, \dots, \matr{R}_n$ and translations $\vectr{t}_1, \dots, \vectr{t}_n$. 
		Depending on how the poses are obtained, they are all relative to a common (world) coordinate frame.
		Sometimes it is useful to have the coordinate frame of the first camera transform be the common frame for all transformations.
		This can be achieved by applying the transformations
		\begin{align}
			\label{eq:relative_rotation_conversion}
			\begin{split}
				\matr{R}_{i}^\prime 	&= \matr{R}_{1}^{\top} \matr{R}_{i} \\
				\vectr{t}_{i}^\prime 	&= \matr{R}_{1}^{\top} (\vectr{t}_i - \vectr{t}_1)
			\end{split}
		\end{align}
%		\begin{equation}
%			\matr{R}_{i \rightarrow 1} 	= \matr{R}_{\text{w} \rightarrow 1} \matr{R}_{i \rightarrow \text{w}}
%										= \matr{R}_{1 \rightarrow \text{w}}^{-1} \matr{R}_{i \rightarrow \text{w}}
%										= \matr{R}_{1 \rightarrow \text{w}}^{\top} \matr{R}_{i \rightarrow \text{w}}
%		\end{equation}
%		\begin{equation}
%		\vectr{t}_{i \rightarrow 1} = \matr{R}_{1 \rightarrow \text{w}}^{\top} (\vectr{t}_i - \vectr{t}_1)
%		\end{equation}
		to get the new relative rotation $\matr{R}_{i}^\prime$ and relative translation $\vectr{t}_{i}^\prime$.
		Note that for $i = 1$ we obtain $\matr{R}_{1}^\prime = \matr{I}$ and $\vectr{t}_{1}^\prime = \vectr{0}$.
		
		\noindent\rule{2cm}{0.4pt}
		\todo{make good transition}
		
		So far, we have seen rotations represented as matrices. 
		However, it is redundant to describe rotations with nine numbers when in fact there are only three degrees of freedom for a three-dimensional rotation, i.e. two degrees for the orientation of the axis and one for the angle of rotation around the axis.
		It is also not straightforward to interpolate rotations in $SO(3)$, or to find the best approximation for a matrix that lies outside of $SO(3)$.\todo{citation needed}
		
		Below we discuss and compare three other popular representations used to describe rotations.
		A brief overview is also shown in table~\ref{tbl:comparison_representations_of_rotations}.
		\begin{table}
			\small
			\begin{center}
				\begin{tabular}{|l|p{2.5cm}|p{2.5cm}|p{2.5cm}|p{2.5cm}|}
					\hline
					& Euler angles & Matrix & Axis-angle & Unit quaternion \\ \hline
					Typical use case & Robotics, Avionics & Computer Graphics, Physics & Intermediate representation & Computer Graphics, Physics \\ \hline
					Constraint & None & Orthonormality & Unit norm axis & 4D unit sphere \\ \hline
					Disadvantage & Gimbal lock & Numerically unstable & & Less intuitive \\ \hline
					Interpolation & hard & hard & hard & easy \\ \hline
					Identity rotation & ambiguous & unique & ambiguous &unique \\ 
					\hline
				\end{tabular}
			\end{center}
			\label{tbl:comparison_representations_of_rotations}
			\caption[short text]{text}
		\end{table}
		
		\paragraph{Euler Angles}
		
		\paragraph{Axis-Angle}
		
		\paragraph{Quaternions}
		Quaternions are four-dimensional numbers.
		They are an extension of the complex numbers\footnote{A more appropriate name is \emph{compound numbers}.\todo{disputable}}.
		Formally, a quaternion is defined as $\vectr{q} = w + \vectr{i}x + \vectr{j}y + \vectr{k}z$ where $w, x, y, z \in \mathbb{R}$ and $\vectr{i},\vectr{j}$ and $\vectr{k}$ are the basis elements of the quaternion space.
		The quaternion can also be written as a tuple $\vectr{q} = (w, x, y, z) \in \mathbb{R}^4$ by using the notation
		\begin{equation}
			\vectr{1} = 
			\begin{bmatrix}
				1 \\ 
				0 \\ 
				0 \\ 
				0
			\end{bmatrix},\quad
			\vectr{i} = 
			\begin{bmatrix}
				0 \\ 
				1 \\ 
				0 \\ 
				0
			\end{bmatrix},\quad
			\vectr{j} = 
			\begin{bmatrix}
				0 \\ 
				0 \\ 
				1 \\ 
				0
			\end{bmatrix},\quad
			\vectr{k} =
			\begin{bmatrix}
				0 \\ 
				0 \\ 
				0 \\ 
				1
			\end{bmatrix}.
		\end{equation}
		But quaternions are not just tuples of numbers. 
		They are equipped with a separate addition and multiplication.
		Addition is simply carried out element-wise, whereas multiplication makes use of the distributive rule and the property
		\begin{equation}
			\vectr{i}^2 = \vectr{j}^2 = \vectr{k}^2 = \vectr{i}\vectr{j}\vectr{k} = -1.
		\end{equation}
		Furthermore, the norm of a quaternion is defined as
		\begin{equation}
			\lVert \vectr{q} \rVert = \sqrt{w^2 + x^2 + y^2 + z^2}.
		\end{equation}
		
		In this thesis, we are particularly interested in the subset of quaternions called the \emph{unit quaternions}. These are all the quaternions with unit norm, i.e. $\lVert \vectr{q} \rVert = 1$. 
		The quaternions of this form describe 3D rotations and can be parameterized by
		\begin{equation}
			\cos\left(\tfrac{\theta}{2}\right) + 
			\sin\left(\tfrac{\theta}{2}\right) \left(a_1 \vectr{i} + a_2 \vectr{j} + a_3 \vectr{k}\right),
		\end{equation}
		where $\theta \in \mathbb{R}$ is the angle of rotation around a unit-length axis $\vectr{a} = (a_1, a_2, a_3) \in \mathbb{R}^3$.
		\todo{explain that quaternions are not unique representation of rotations, and that negative axis-angle is represented by the same quaternion}
	
	\section{PyTorch}