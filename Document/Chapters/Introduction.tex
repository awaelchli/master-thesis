\chapter{Introduction}

	\section{Relative Camera Pose}
		Equations to transform pose in world coordinates to relative pose with respect to the coordinate frame of the first camera.
		\begin{equation}
			\matr{R}_{2 \rightarrow 1} 	= \matr{R}_{\text{w} \rightarrow 1} \matr{R}_{2 \rightarrow \text{w}}
										= \matr{R}_{1 \rightarrow \text{w}}^{-1} \matr{R}_{2 \rightarrow \text{w}}
										= \matr{R}_{1 \rightarrow \text{w}}^{\top} \matr{R}_{2 \rightarrow \text{w}}
		\end{equation}
		
		\begin{equation}
			\matr{R}_{2 \rightarrow 1} = \matr{R}_{1 \rightarrow \text{w}}^{\top} (\matr{t}_2 - \matr{t}_1)
		\end{equation}


	\section{Rotation Metrics}
	\cite{huynh2009metrics}
	
	\section{Two-Frame Structure from Motion}
	
	\section{Multiview Structure from Motion}
	
	\section{Quaternions}
		Quaternions are four-dimensional numbers.
		They are an extension of the complex numbers\footnote{A more appropriate name is \emph{compound numbers}.\todo{disputable}}.
		Formally, a quaternion is defined as $\vectr{q} = w + \vectr{i}x + \vectr{j}y + \vectr{k}z$ where $w, x, y, z \in \mathbb{R}$ and $\vectr{i},\vectr{j}$ and $\vectr{k}$ are the basis elements of the quaternion space.
		The  can also be written as a tuple $\vectr{q} = (w, x, y, z) \in \mathbb{R}^4$ by using the notation
		\begin{equation}
			\vectr{1} = 
			\begin{bmatrix}
				1 \\ 
				0 \\ 
				0 \\ 
				0
			\end{bmatrix},\quad
			\vectr{i} = 
			\begin{bmatrix}
				0 \\ 
				1 \\ 
				0 \\ 
				0
			\end{bmatrix},\quad
			\vectr{j} = 
			\begin{bmatrix}
				0 \\ 
				0 \\ 
				1 \\ 
				0
			\end{bmatrix},\quad
			\vectr{k} =
			\begin{bmatrix}
				0 \\ 
				0 \\ 
				0 \\ 
				1
			\end{bmatrix}.
		\end{equation}
		But quaternions are not just tuples of numbers. 
		They are equipped with a separate addition and multiplication.
		Addition is simply carried out element-wise, whereas multiplication makes use of the distributive rule and the property
		\begin{equation}
			\vectr{i}^2 = \vectr{j}^2 = \vectr{k}^2 = \vectr{i}\vectr{j}\vectr{k} = -1.
		\end{equation}
		
		
	
	\section{PyTorch}