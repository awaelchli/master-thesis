\chapter{Prior Work}

	\todo{Add text that introduces the chapter}

    \section{04/2017 - SfM-Net}
    
        \cite{SFMNET} implement a deep learning approach to structure from motion. 
        Their architecture consists of two subnetworks:
        \begin{itemize}
            \item \textbf{Structure }
                \\
                Learns per-frame depth.
                The input is a single frame. 
                The output of the CNN is a cloud of 3D points, one for each pixel value in the input image.
            \item \textbf{Motion}
                \\
                The input is a pair of frames.
                The CNN computes a set of $K$ segmentation masks for moving objects. 
                Using these masks, each pixel is assigned to an object specific transformation given by a rotation and a translation.
                In addition, camera rotation and translation are computed using the features of the inner layers.
        \end{itemize}
        Given a pair of images, the forward operation of the network works as follows:
        \begin{enumerate}
            \item The structure network computes the point cloud for the first frame.
            \item The motion network computes the object transformations as well as the camera transformation using both frames.
            \item The point cloud is transformed using the learned object transformations and masks.
            \item The transformed 3D points are re-projected to 2D using the learned camera transformation between the two frames.
        \end{enumerate}
        The transformed point cloud corresponds to the depth of the second frame.
        Optical flow can be computed directly from the re-projected points.
        
        The authors of the paper propose various modes of supervision to evaluate the architecture and to handle ambiguities in the reconstruction due to the ill-posed problem:
        \begin{itemize}
            \item \textbf{Self-supervision}
                \\
                No ground truth is given.
                The loss is defined by the brightness constancy constraint of the second frame warped to the first frame using the predicted optical flow.
                For this mode, they use the {KITTI} 2012/15 datasets
            \item \textbf{Depth}
                \\
                Ground truth is given in the form of depth for each pixel.
                This can be acquired for example by a Kinect sensor.
                They use the {RGB-D SLAM} dataset for ground truth depth.
                This helps to improve camera motion estimation. 
            \item \textbf{Camera motion}
                \\
                Camera motion is given as ground truth in form of a rotation and translation matrix.
                The relative transformation between predicted and ground truth transformation is 
            \item \textbf{Optical flow and object motion}
                \\
                They use this type of supervision with the MoSeg dataset which contains ground truth segmentation for each frame.
                This dataset contains more non-rigid body transformation.
                They evaluate the quality of the object motion mask by Intersection over union (IoU).
        \end{itemize}
        
    \section{07/2016 - Unsupervised CNN for Single View Depth Estimation}
    
        The work of \cite{garg2016} implements a autoencoder on stereo pair images to predict depth from a single image.
        Their architecture consist of the following parts.
        \begin{itemize}
        	\item \textbf{Encoder}
        		\\
        		Takes a single image as input. 
        		At training time, this is the left image of a stereo pair.
        		The output is the predicted disparity map (scaled inverse depth).
        	\item \textbf{Decoder}
        		\\
        		The decoder is only used at training time.
        		It takes two inputs: The predicted disparities from the encoder and the right image of the stereo pair.
        		The right image is warped using the displacements and compared to the encoder input (left image) using the color constancy error (photometric error).
        \end{itemize}
        Since the stereo pair is rectified, the disparity is a displacement along the scanline of the images.
        Thus the decoder implements a simple geometric transformation that does not need to be learned.
        At test time, only the encoder network with a single image as input is used.
        
        As noted by the author, there are standard stereo algorithms that produce disparity maps.
        However, these methods can not deal with distortions such as lens flare, motion blur, shadows, etc.
        The idea is that a neural network could learn to deal with such problems that occur in natural images.
        
        The dataset they use is {KITTI} which they augment by random crops, color channel scaling, and flipping the images.
        
        \section{05/2017 - A Survey on Structure from Motion}
        
        	\cite{survey2017} give a very good and concise definition of structure from motion:
        	
        	\say{The structure from motion (SfM) problem in computer vision is the problem of recovering the three-dimensional (3D) structure of a stationary scene from a set of projective measurements, represented as a collection of two-dimensional (2D) images, via estimation of motion of the cameras corresponding to these images.}
        	
        	The three basic steps of SfM are:
        	\begin{itemize}
        		\item Feature detection, extraction and matching
        		\item Camera motion estimation
        		\item Recovery of 3D structure
        	\end{itemize}
        	
        	\paragraph{Bundle adjustment} 
        		This technique simply considers to minimize the re-projection error of the unknown 3D points and camera matrices.
        		The re-projection error is formulated as the euclidean distance between the known image coordinates and the projection of the unknown points using the unknown camera matrices.
        		As stated by \cite{survey2017}, the problem is non-convex and in practice, common optimization algorithms achieve only a poor local minimum.
        		
        	\paragraph{The eight point algorithm}
        	
        	\paragraph{Factorization methods}
        		The first method proposed by \cite{tomasi1992factorization} only works with an orthographic camera model.
        		A measurement matrix $W \in \mathbb{R}^{2n \times m}$ is defined as
        		\begin{equation}
        			W =
        			\begin{bmatrix}
        			x_{11} & \cdots & x_{1m} \\ 
        			\vdots & \ddots & \vdots \\ 
        			x_{n1} & \cdots & x_{nm} \\ 
        			y_{11} & \cdots & y_{1m} \\ 
        			\vdots & \ddots  & \vdots \\ 
        			y_{n1} & \cdots & y_{nm}
        			\end{bmatrix}. 
        		\end{equation}
        		It contains the 2D coordinates of the $m$ orthographically projected 3D points for each of the $n$ cameras.
        		Under the assumption that the origin of global coordinate system is located at the center of the point cloud, the matrix $W$ can be factorized into
        		\begin{equation}
        			W = RS,
        		\end{equation}
        		where $R \in \mathbb{R}^{2n \times 3}$ contains the orientation vectors of each camera and $S \in \mathbb{R}^{3 \times m}$  are the 3D points.
        		As described by \cite{tomasi1992factorization}, this factorization can be achieved using the singular value decomposition (SVD) as $W = U \varSigma V^\top$ and deriving $R$ and $S$ from $U$, $V$ and $\varSigma$.
        		\todo{Add more details? Less details?}
        		This factorization method was later extended by \cite{sturm1996factorization} for projective cameras.
        		
        		
        	
        	
        	
        
        
        
        